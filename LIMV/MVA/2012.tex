\section*{2012}
\vspace{-.5cm}
\hrulefill \smallskip\\
\ques{5}{a}{12} $X$ follow $N_3\left(0,\Sigma\right)$ where $\displaystyle \mathbf{\Sigma} =  \begin{bmatrix} 1 & \rho & 0 \\ \rho & 1 & \rho \\ 0 & \rho & 1 \end{bmatrix}$.\\ Find $\rho$ such that $(X_1+X_2+X_3)$ and $(X_1-X_2 -X_3)$ are independent.
\myline
\ques{6}{a}{20} Suppose that $X$ follow a $p$-variate normal distribution with co-variance matrix \[ \mathbf{\Sigma} = \begin{bmatrix} \sigma^2 & \rho\sigma^2 & \rho\sigma^2 & \cdots & \rho\sigma^2 \\
\rho\sigma^2 & \sigma^2 & \rho\sigma^2 & \cdots & \rho\sigma^2 \\
\rho\sigma^2 & \rho\sigma^2 & \sigma^2 & \cdots & \rho\sigma^2 \\
\vdots & \vdots & \vdots & \ddots & \vdots \\ 
\rho\sigma^2 & \rho\sigma^2 & \rho\sigma^2 & \cdots & \sigma^2 \\
\end{bmatrix} \] Obtain the expression for the multiple correlation coefficient $R_{1\cdot23\ldots p}$.
\myline
\ques{6}{c}{20} Suppose $\mathbf{X} = [X_1,X_2,X_3]^\prime$ has a covariance matrix
\[ \mathbf{\Sigma} = \begin{bmatrix} 1 & -2 & 0 \\ -2 & 5 & 0 \\ 0 & 0 & 2 \end{bmatrix}.\] Obtain the first two principle components of $\mathbf{X}$.
\myline
\ques{7}{b}{20} Bring out the relation between Hotelling's $T^2$ and Mahalanobis $D^2$. Explain how $T^2$ can be interpreted as an extension of the student's $t$ statistic. Describe the utility of Fisher's discriminant function in classification problems.