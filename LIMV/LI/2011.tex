\section*{2011}
\vspace{-.5cm}
\hrulefill \smallskip\\
\ques{5}{a}{12} The p levels of a  factor A are the only levels the experimenter is interested in, while the q levels of the factor B chosen form a random sample from the totality of levels available. only one observation per cell $A_i\times B_j$, $i=1(1)p$, $j=1(1)q$, is taken on the response variable. Write down the appropriate linear model, the hypotheses to be tested, the test statistics and the level-$\alpha$ critical regions.
\myline
\ques{5}{b}{12} Assuming the model
\[ \begin{aligned} x_i &= m_1 + e_{1i} \\
    y_i &= m_2 + e_{2i} \\
    z_i &= m_1 + m_2 + e_{3i}, \enskip i=1(1)n
\end{aligned}\] where $e_{1i}$, $e_{2i}$, $e_{3i}$ are the errors, obtain least square estimates $m_1$ and $m_2$.
\myline
\ques{5}{e}{12} Suppose, with usual notation
\[ \begin{aligned} r_{12} &=0.36  &r_{13} &=0.29   \\
r_{14} &= -0.62  &r_{23} &= 0.41  \\
r_{24} &= -0.24 \quad &r_{34} &= -0.52
\end{aligned}\] For predicting $x_1$, the variable $x_4$ has already been included in the regression equation. Which one of $x_2$ and $x_3$ is worth including in the regression equation in addition to $x_4$?
\myline
\ques{6}{a}{20} Give the outline of procedure, using ANOVA technique, for testing if the response variable $y$ is dependent on a fixed set of variables $x_1,x_2,\dotsc,x_k$, based on an observed set of data on these variables.
\myline
\ques{8}{a}{20} Does the observed correlation between two variables $X_1$ and $X_2$ always imply causal relationship between them? Give an example. Define partial correlation coefficient $r_{12\cdot3}$ more suitable for throwing light on the net correlation between  $X_1$ and $X_2$? \smallskip\\
Based on the observed data set $x_{ij}, i = 1,2,3$ and $j=1(1)n$, derive the expression for $r_{12\cdot3}$.