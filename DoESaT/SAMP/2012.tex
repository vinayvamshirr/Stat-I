\section*{2012}
\vspace{-.5cm}
\hrulefill \smallskip\\
\ques{5}{d}{12} Distinguish between simple random sampling with and without replacement. Write down the expression for the number of samples and probability that a specified individual is included in the sample under both the sampling schemes SRSWR and SRSWOR.
\myline
\ques{7}{a}{20} Let $N=3$, $n =2$ and the possible distinct samples are $s_1=(1,2)$, $s_2=(1,3)$, $s_3=(2,3)$. If SRSWOR is adopted, write down $p(s_i)$ for $i =1,2,3$. If\[ t= \begin{cases} y_{1/2} + y_{2/2} \&\text{if $s_1$ occurs}\\
y_{1/2} + 2y_{3/3} \enskip &\text{if $s_2$ occurs}\\
y_{2/2} + y_{3/3} &\text{if $s_3$ occurs}, \end{cases}\] Examine whether $t$ is unbiased for the population mean. Also find the expression for $v(t)$, if unbiased.
\myline
\ques{8}{a}{20} What are inclusion probabilities? USe this concept to define the Horvitz-Thomson estimator. If $\alpha_k$ and $\alpha_{kl}$ denote the first and second order inclusion probabilities under an arbitrary ordered design of fixed size, show that \begin{tasks} \task $\displaystyle \sum_1^N\alpha_k = v$  and
\task $\displaystyle \sum\sum_{k\neq l}^N \alpha_{kl} = v(v-1)$
\end{tasks} where $v = E(v(s))$ is the expected effective sample size.
\myline
\ques{8}{b}{20} IF $V_{ran}$, $V_{prop}$, $V_{opt}$ respectively denote the variance of the estimator of the population mean under SRSWOR, stratified sampling under proportional allocation and Neyman allocation show that
\[ V_{ran} > V_{prop} > V_{opt} .\] 