\section*{2011}
\vspace{-.5cm}
\hrulefill \smallskip\\
\ques{5}{c}{12} 6 days are to be selected from the 366 calendar days of 2012 for recording the maximum daily temperature in a metropolitan city. Using the following set of random numbers, select these days, mentioning the month and the date for each, and giving the procedure in detail
\begin{table}[htbp]
    \centering
    \begin{tabular}{*{5}{c}}
        6503 & 0085 & 3822 & 2193 & 5392 \\
        4635 & 0495 & 3296 & 1348 & 
    \end{tabular}
\end{table}
\ques{6}{b}{20} Determine the sample size $n$ such that the estimated value, based on this sample, of the mean $\text{SO}_2$ level in the air in 1000 petrol pumps in a large town, differs from the true value by at most $5\mu \text{g/m}^3$ with probability 0.95, assuming the standard deviation to be $35\mu\text{g/m}^3$.[2.5\% and 5\% points of a standard normal variate are 1.960 and 1.645 respectively.]
\myline
\ques{7}{b}{20} The total population size of a district in 2001 census is available. The population sizes of a random sample of $n$ villages from the district individually in both 2001 and 2011 censuses are also available. Suggest a suitable estimator of the total population of the district in 2011, based on these figures using the ration method of estimation. \smallskip \\
Derive an expression for the bias of the estimator and deduce the condition under which the bias in negligible.
\myline
\ques{8}{c}{20} In a planned town, there are N blocks each containing $M$ residential buildings. A simple random sample of $n$ blocks is drawn, and the numbers of working women in all the residential buildings belonging to these selected blocks are counted. \smallskip \\
Based on these figures, suggest an unbiased (to be shown) estimator of the total number of working women in the town, and derive the standard error of the estimator.