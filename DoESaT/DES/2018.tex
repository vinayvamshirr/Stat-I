\section*{2018}
\vspace{-.5cm}
\hrulefill \smallskip\\
\ques{5}{c}{10} (i) Distinguish between total and partial confounding in a $2^k$-factorial experiment.\\
(ii) If a $2^3$ factorial experiment is laid in a block of size 4 with four replicates, write down the allocations of degrees of freedom for analyzing the results of such a design when the highest order interaction is totally confounded.
\myline
\ques{5}{e}{10} In a randomized block design with t treatments and r replicates, one observation was found missing. Explain the method of estimating the missing value and perform the analysis of variance.
\myline
\ques{6}{a}{20} What do you understand by Balanced design? Suppose four treatments (1,2,3,4) are laid out in five blocks (I,II,II,IV,V) as given below:
\begin{center}
\begin{tabular}{lccc}
     I & 1 & 1 & 2 \\
     II& 1 & 2 & 2 \\
     III & 3 & 4 & \\
     IV & 3 & 4 & \\
     V & 3 & 4 &
\end{tabular}
\end{center}
 Establish that the above design is unbalanced.
\myline
\ques{7}{b}{15} A $2^5$ experiment involving factors A,B,C,D and E is conducted in the blocks of size 8. The key bloks in the replicates 1 and 2 are respectively given by 
\[ \begin{aligned}
&\text{1, bd, abde, cde, acd, bce, abc, ae and }\\
& \text{1, ac, abd, ade, bcd, cde, be, abce. }    
\end{aligned}\] Identify the confounding effects in replication.
