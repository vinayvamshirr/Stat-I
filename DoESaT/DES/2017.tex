\section*{2017}
\vspace{-.5cm}
\hrulefill \smallskip\\
\ques{5}{e}{10} Explain the concept of balanced incomplete block design (BIBD). What are the conditions for existence of a BIBD? If V={1,2,3,4,5,6,7}, then form a number of blocks, each of order 3. such that each pair of elements in V is contained in exactly one block.
\myline
\ques{6}{c}{15} Distinguish between a factorial experiment and a number of single-factor experiments. What is meant by confounding in a factorial experiment? Why is confounding preferred even at the cost of loss of information on the confounding effects?
\myline
\ques{8}{a}{20} Three sides of an equilateral triangle were measured by 5 persons with the following results :
\begin{center}
    \begin{tabular}{|*{6}{c|}} \hline
        \multirow{2}{*}{Sides} & \multicolumn{5}{c|}{Persons}\\ \cline{2-6}
        & A & B & C & D & E \\ \hline
        a & 5.44 & 5.41 & 5.43 & 5.42 & 5.43 \\ \hline
        b & 5.43 & 5.41 & 5.42 & 5.43 & 5.44 \\ \hline
        c & 5.45 & 5.42 & 5,43 & 5.43 & 5.33 \\ \hline
    \end{tabular}
\end{center}