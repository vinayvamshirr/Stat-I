\section*{2009}
\vspace{-.5cm}
\hrulefill \smallskip\\
\ques{5}{e}{12} Suggest a balanced confounded design of a $2^4$ experiment in 3 replicates, each replicate containing 4 incomplete blocks of size 4 each, retaining full information on the main effects and the 2nd order interactions.

Give the complete lay-out of one such replicate indicating the factorial effects confounded in it.
\myline
\ques{5}{f}{12} Define a BIBD. Show that for a BIBD with parameters v, b, r, k and $\lambda$,
\begin{tasks}[]
    \task $\lambda$(v - 1) = r(k-1)
    \task b \geq v + r -k.
\end{tasks}
\ques{6}{c}{20} Use the following set of random numbers to obtain the layout of a randomized block design with 5 treatments and 4 blocks, giving the outline of the procedure you have followed:
\begin{center}
 \begin{tabular}{*{8}{c}}
    5711 & 7343 & 7539 & 3684 & 9397 & 5335 & 4031 & 1486  \\
    2588 & 3301 & 0533 & 2427 & 3598 & 2580 & 7017 & 9176 
\end{tabular}   
\end{center}
\ques{7}{c}{20} 3 factors, each at 2 levels, are to be tested in a single experiment using r randomized blocks. Using standard notation, give the expressions for the sums of squares due to different factorial effects. Also give the ANOVA table.
\myline
\ques{8}{c}{20} Compare completely Randomized Design, Randomized Block Design and Latin Square Design in terms of flexibility for number of treatments and number of replicates per treatment on the one hand, and error control on the other.