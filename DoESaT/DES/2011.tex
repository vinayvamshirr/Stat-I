\section*{2011}
\vspace{-.5cm}
\hrulefill \smallskip\\
\ques{5}{d}{12} Suggest the confounding scheme of a $2^4$- design to be carried out in 4 replicates each comprising 4 incomplete blocks of size 4, such that full information is retained on the main effects, each of the first-order and second-order interactions is confounded only in one replicate, while the third-order interaction is confounded in two replicates. \smallskip \\ Give the complete layout of one such replicate indicating the factorial effects confounded in it.
\myline
\ques{6}{c}{20} Introduce the concepts of confounding, total confounding, partial confounding and balanced confounding, partial confounding and balanced confounding in the context of a factorial design. \smallskip \\
A $3^2$-design has been conducted using 4 replicates each comprising 3 incomplete blocks of size 3 each. In two of the replicates, $AB$ has been confounded, while in the remaining two, $AB^2$ has been confounded. \smallskip \\ 
Write down the ANOVA table. Indicate how the sums of squares due to different factorial effects will be calculated.
\myline
\ques{7}{c}{20} Construct a pair of orthogonal Latin squares with symbols $A$, $B$, $C$ and $\alpha$, $\beta$, $\gamma$ respectively. Superimpose one onto the other. Numbering the small squares from 1 to 9, use this inomplete block design with parameters $v=9$, $b=12$, $r=4$, $k=3$ and $\lambda = 1$.
