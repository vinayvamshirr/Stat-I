\section*{2013}
\vspace{-.5cm}
\hrulefill \smallskip\\
\ques{4}{a}{20} Let $N$ be the number of observatins required by the sequential probability ratio test (SPRT) for testing \\
$\text{H}_0 : X \sim f(x, \theta_0) $ against $\text{H}_0 : X \sim f(x, \theta_1), \theta_1 \neq \theta_0$. Stating appropriate assumptions, show that $P(N<\infty) = 1$ at $\theta = \theta_0,\theta_1$. \\
When $X\sim N(\theta,1)$, for testing $\text{H}_0 : \theta = 0 $ against $\text{H}_1 : \theta = 1 $, derive the expression of the percentage saving in sample sizes under $\text{H}_0 $ based on SPRT over the fixed sample size based MP test. Take the strength of each procedure as $(\alpha,\beta)$.