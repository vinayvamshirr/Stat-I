\section*{2009}
\vspace{-.5cm}
\hrulefill \smallskip\\
\ques{1}{d}{12} State the invariance property of maximum likelihood estimator (m.l.e.). Use this to obtain the m.l.e of $\dfrac{1}{\theta}$ in sampling from 
\[ f(x;\theta) = 
\begin{cases}
\dbinom{m}{x}\theta^x (1-\theta)^{m-x} &,x=0(1)m, \\
\quad0 &,\text{otherwise}.
\end{cases}
\]
\myline
\ques{3}{b}{20} For the pdf 
\[ f(x;\theta) = 
\begin{cases}
1/ \big[ \pi \{1 + (x-\theta)^2 \} \big], &-\infty < x < \infty,\\
\quad0 , &\text{elsewhere} ,
\end{cases} \] derive Rao-Cramer lower bound of variance of an unbiased estimator of $\theta$.